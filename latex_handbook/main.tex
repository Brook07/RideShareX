\documentclass[11pt,a4paper]{article}
\usepackage[utf8]{inputenc}
\usepackage[margin=0.8in]{geometry}
\usepackage{hyperref}
\usepackage{listings}
\usepackage{xcolor}
\usepackage{tcolorbox}
\usepackage{enumitem}

\definecolor{codegray}{rgb}{0.95,0.95,0.95}
\lstset{
    backgroundcolor=\color{codegray},
    basicstyle=\ttfamily\small,
    breaklines=true,
    frame=single,
    numbers=none
}

\title{\textbf{RideShareX Full-Stack Handbook}\\[0.3cm]\large Quick Reference Guide}
\author{Computer Science Learning Resource}
\date{\today}

\begin{document}
\maketitle
\tableofcontents
\newpage

%=====================================
\section{System Design Overview}
%=====================================

\subsection{Frontend vs Backend}
\begin{tcolorbox}[colback=blue!5]
\textbf{Frontend:} Runs in browser. Shows UI, captures input, calls APIs.\\
\textbf{Backend:} Runs on server. Stores data, enforces rules, handles security.
\end{tcolorbox}

\subsection{Client-Server Flow}
\begin{verbatim}
Browser (Client) --HTTP Request--> Backend Server --Query--> Database
Browser          <--JSON Response-- Backend Server <--Data--- Database
\end{verbatim}

\subsection{What Happens on "Book Now" Click}
\begin{enumerate}[noitemsep]
    \item User fills form, clicks button
    \item Frontend validates input
    \item Frontend sends POST /api/bookings with JWT token
    \item Backend verifies token, validates data
    \item Backend saves booking to database
    \item Backend returns JSON response
    \item Frontend updates UI
\end{enumerate}

%=====================================
\section{Frontend (React)}
%=====================================

\subsection{Core Concepts}
\begin{itemize}[noitemsep]
    \item \textbf{Component:} Reusable UI piece that returns JSX
    \item \textbf{JSX:} HTML-like syntax in JavaScript
    \item \textbf{Props:} Data passed from parent to child (read-only)
    \item \textbf{State:} Data owned by component (can change)
\end{itemize}

\subsection{Key Hooks}
\begin{lstlisting}[language=JavaScript]
// useState - manage component state
const [loading, setLoading] = useState(false);

// useEffect - run side effects (API calls)
useEffect(() => {
  fetchVehicles();
}, []); // [] = run once on mount
\end{lstlisting}

\subsection{API Call Pattern}
\begin{lstlisting}[language=JavaScript]
const response = await fetch('/api/bookings', {
  method: 'POST',
  headers: {
    'Content-Type': 'application/json',
    'Authorization': `Bearer ${token}`
  },
  body: JSON.stringify({ vehicleId, pickupDate, dropDate })
});
const data = await response.json();
\end{lstlisting}

%=====================================
\section{Backend (Node.js + Express)}
%=====================================

\subsection{Core Concepts}
\begin{itemize}[noitemsep]
    \item \textbf{Route:} Maps URL + HTTP method to handler
    \item \textbf{Controller:} Handles request/response
    \item \textbf{Middleware:} Runs before controller (auth, validation)
    \item \textbf{Service:} Business logic layer
\end{itemize}

\subsection{Basic Express Setup}
\begin{lstlisting}[language=JavaScript]
const express = require('express');
const app = express();
app.use(express.json()); // parse JSON body

app.get('/api/vehicles', (req, res) => {
  res.json({ vehicles: [...] });
});

app.listen(5000);
\end{lstlisting}

\subsection{REST API Rules}
\begin{tabular}{|l|l|l|}
\hline
\textbf{Method} & \textbf{Action} & \textbf{Example} \\
\hline
GET & Read & GET /api/vehicles \\
POST & Create & POST /api/bookings \\
PUT & Update & PUT /api/bookings/:id \\
DELETE & Delete & DELETE /api/bookings/:id \\
\hline
\end{tabular}

\vspace{0.3cm}
\subsection{Status Codes}
\begin{tabular}{|l|l|}
\hline
200 & OK (success) \\
201 & Created (POST success) \\
400 & Bad Request (invalid input) \\
401 & Unauthorized (no/bad token) \\
403 & Forbidden (no permission) \\
404 & Not Found \\
500 & Server Error \\
\hline
\end{tabular}

%=====================================
\section{Database (MongoDB)}
%=====================================

\subsection{SQL vs NoSQL}
\begin{tabular}{|l|l|}
\hline
\textbf{SQL} & \textbf{NoSQL (MongoDB)} \\
\hline
Tables with rows & Collections with documents \\
Fixed schema & Flexible schema \\
Joins & References + populate \\
\hline
\end{tabular}

\subsection{RideShareX Schemas (Summary)}
\begin{lstlisting}[language=JavaScript]
// User
{ name, email, password, role: 'user'|'owner' }

// Vehicle
{ owner: ObjectId, title, dailyRate, images, location }

// Booking
{ vehicle: ObjectId, user: ObjectId, owner: ObjectId,
  pickupDate, dropDate, totalPrice,
  status: 'pending'|'approved'|'rejected', expiresAt }
\end{lstlisting}

\subsection{Common Operations}
\begin{lstlisting}[language=JavaScript]
// Create
await Booking.create({ vehicle, user, ... });

// Read
await Vehicle.find({ dailyRate: { $lte: 100 } });
await Booking.findById(id).populate('vehicle');

// Update
await Booking.findByIdAndUpdate(id, { status: 'approved' });

// Delete
await Vehicle.findByIdAndDelete(id);
\end{lstlisting}

%=====================================
\section{Authentication (JWT)}
%=====================================

\subsection{Flow}
\begin{enumerate}[noitemsep]
    \item User sends email/password to POST /api/auth/login
    \item Backend verifies, returns JWT token
    \item Frontend stores token (localStorage)
    \item Frontend sends token in every request header
    \item Backend middleware verifies token, attaches user to req
\end{enumerate}

\subsection{JWT Structure}
\begin{verbatim}
header.payload.signature
Payload: { id: "user123", role: "user", exp: timestamp }
\end{verbatim}

\subsection{Auth Middleware}
\begin{lstlisting}[language=JavaScript]
const protect = async (req, res, next) => {
  const token = req.headers.authorization?.split(' ')[1];
  if (!token) return res.status(401).json({ msg: 'No token' });
  
  const decoded = jwt.verify(token, process.env.JWT_SECRET);
  req.user = await User.findById(decoded.id);
  next();
};

// Usage
app.post('/api/bookings', protect, createBooking);
\end{lstlisting}

%=====================================
\section{Payment Flow (eSewa)}
%=====================================

\begin{enumerate}[noitemsep]
    \item Owner approves booking
    \item User clicks "Pay Now"
    \item Backend creates payment record, returns eSewa params
    \item Frontend redirects to eSewa
    \item User pays on eSewa
    \item eSewa redirects back with success/failure
    \item Backend verifies via webhook, updates payment status
\end{enumerate}

%=====================================
\section{Time-Based Logic}
%=====================================

\subsection{Auto-Expire Bookings}
\begin{lstlisting}[language=JavaScript]
// When creating booking
expiresAt: new Date(Date.now() + 5 * 60 * 60 * 1000) // 5 hours

// Cron job (runs every 10 min)
cron.schedule('*/10 * * * *', async () => {
  await Booking.updateMany(
    { status: 'pending', expiresAt: { $lt: new Date() } },
    { status: 'expired' }
  );
});
\end{lstlisting}

%=====================================
\section{Best Practices Checklist}
%=====================================

\begin{itemize}[noitemsep]
    \item[$\square$] Validate on both frontend AND backend
    \item[$\square$] Hash passwords (bcrypt)
    \item[$\square$] Use HTTPS in production
    \item[$\square$] Set JWT expiration
    \item[$\square$] Handle errors with try/catch
    \item[$\square$] Use environment variables for secrets
    \item[$\square$] Separate concerns: routes $\rightarrow$ controllers $\rightarrow$ services
    \item[$\square$] Use meaningful HTTP status codes
    \item[$\square$] Sanitize user input
    \item[$\square$] Log errors for debugging
\end{itemize}

%=====================================
\section{Project Structure}
%=====================================

\begin{verbatim}
frontend/
  src/
    components/   # Reusable UI
    pages/        # Route pages
    context/      # Auth context
    services/     # API calls

backend/
  config/         # DB connection
  controllers/    # Request handlers
  middleware/     # Auth, validation
  models/         # Mongoose schemas
  routes/         # API routes
  server.js       # Entry point
\end{verbatim}

\end{document}
